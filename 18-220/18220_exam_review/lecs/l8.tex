\subsection*{Lecture 8}
Inductor design, transformers.

\begin{enumerate}
  \item You have probably heard ``parasitic capacitance'' and ``parasitic inductance''
  many times now in this course. What do they mean? Why do we call them ``parasitic''?
  \item What is mutual inductance?
  \item What are transformers and why are they useful?
  \item What is the ideal transformer model? Draw it.
  \item How do we find the equivalent resistance of a resistor on the secondary side of a tranformer?
  \item There are several forms of non-idealness for a transformer.
  In our class, we consider
  \begin{itemize}
    \item Imperfect ($k\neq 1$)
    \item Non-ideal (general)
  \end{itemize}
  for each, explain their implications and how we solve circuit problems with these non-ideal properties.
\end{enumerate}

\subsection*{Equations}
Transformer equivalent impedance
\begin{figure}[H]
  \def\offsetX{6}
  \def\offsetY{-1}
  \centering
  \begin{circuitikz}
    \draw (0, 0) node [transformer core](T){};
    \draw
      (T.A1) to[R=$R_1$] ++(-2, 0)
      to[american voltage source, l_=$V_1$]
      ($(T.A2) + (-2, 0)$) node[sground] {}
      -- (T.A2)
      (T.B1) to[short] ++(2, 0)
      to[R=$R_2$]
      ($(T.B2) + (2, 0)$)
      -- (T.B2)
    ;

    \draw
      (4.5, 0) node {$\Rightarrow$}
    ;

    \draw
      (\offsetX, \offsetY) node[sground] {}
      to[american voltage source, V=$V_1$, invert]
      ++(0, 2) to[R=$R_1$] ++(2, 0)
      to[R=$\pa{\frac{N_1}{N_2}}^2R_2$]
      ++(0, -2) -- (\offsetX, \offsetY)
    ;
  \end{circuitikz}
\end{figure}