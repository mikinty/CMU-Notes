\subsection*{Lecture 3}
We learn about diodes and MOSFETs.

\begin{enumerate}
  \item A diode is made from a P and N type. Explain how these two regions build a diode.
  \item Draw the I-V curve for a diode, and label the different regions.
  \item We model a diode has having an exponential relationship between the current and the voltage
  after it is on, that is 
  \begin{equation*}
    I \propto e^V,
  \end{equation*}
  but this means that if we were to turn up the $V$, the current could
  potentially be very large. What do you expect to happen in real life if you
  were to turn the voltage up very high across a diode, assuming you have no
  current limiting in the power supply?

  \label{l3:large_current}

  \item In light of question
  \ref{l3:large_current},
  how do we modify a pure diode circuit to help limit the current?

  %% MOSFET INTRO
  \item What are the 3 parts of a MOSFET?
  \item Write down the MOSFET equations and their operating regions. (You
  don't have to memorize these, but you probably should or else you will be
  too slow on exams)
  \item Draw the $\IDS-\VDS$ curve for a MOSFET, with varying $\VGS$.
  \item Draw the $\IDS-\VGS$ curve for a MOSFET, labeling $\VTH$.
  \item What does it mean for a MOSFET to be ``on''?
  \item Why do we say there is no current through the gate of the MOSFET?
  \item Why do we say the triode region for a MOSFET behaves \textit{linearly}?

  %% LAB
  \item How many volts is a digital high pin when it is on in an Arduino?
  \item In the lab, why can't you see the IR diode turn on and off? 
\end{enumerate}

\subsection*{Equations}
The diode equation is
\begin{equation}
  \iD = \IS \pa{e^{\vD/\VT} - 1}.
\end{equation}
The N-MOSFET equations are 
\begin{align*}
  \ID &= \frac{1}{2}\Kn \pa{\VGS - \VTH}^2 \tag{Saturation, $\VDS > \VGS - \VTH$}\\
  \ID &= \frac{1}{2}\Kn \pa{2\pa{\VGS - \VTH}\VDS - \VDS^2} \tag{Triode, $\VDS < \VGS - \VTH$}
\end{align*}
