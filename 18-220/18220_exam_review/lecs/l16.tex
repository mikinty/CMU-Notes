\subsection*{Lecture 16/17}
There's some review on bode plots but the main focus for 
this one should be op-amp limitations and how feedback comes in 
for analysis of non-ideal opamps. This is also a meaty topic 
for exam questions.

\begin{enumerate}
  \item Describe the assumptions for an ideal opamp.
  \item Why would the opamp behavior change with temperature variation?
  \item Why do we say an ideal op-amp is like a voltage controlled voltage source (VCVS)?
  \item Why do we not want positive gain at \SI{180}{\degree}?
  \item Draw a general feedback diagram with labels, and derive the equation for $\vo/\vi$
  \item Describe the process for finding the following. Draw diagrams too!
  \begin{enumerate}
    \item Loop gain
    \item Open loop gain
    \item Closed loop gain
    \item Write the equation that relates LG, OLG, CLG (Black's formula)
  \end{enumerate}
  \item How can compensation capacitors help with stability?
  \item Write the feedback functions for a Butterworth Filter
  \begin{figure}[H]
    \centering
    \begin{circuitikz}
      \draw (0,0) node[op amp, yscale=-1] (opamp) {}
      (opamp.-) to[short] ++(0,-1) node(ON) {}
      ;

      \draw (opamp.+) 
      to[short] ++(-1,0) node(V2) {}
      to[C=$C_2$] ++(-1.5,0) node(V1) {}
      to[R=$R_1$] ++(-2,0)
      node[left] {$\vi$}
      ;
      
      \draw 
      (V1) 
      to[R=$R_2$] ++(0,1.5)
      -| (opamp.out)
      (V1)
      to[C=$C_1$] ++(0,-3)
      to[short] ($(V2) + (0,-3)$)
      node (GN) [sground] {}
      (V2)
      to[R=$R_3$] (GN)
      ;

      \draw
      (GN)
      to[short] (opamp.- |- 1, -2.5)
      to[R=$R_a$] (ON)
      to[R=$R_b$] (opamp.out |- 1, -1.5)
      to[short] (opamp.out)
      to[short] ++(0.5, 0)
      node[right] {$\vo$}
      ;
    \end{circuitikz}
    \caption{Butterworth Filter}
    \label{l16:butterworth}
  \end{figure}
  \begin{enumerate}
    \item OLG
    \item LG
    \item CLG
    \item Ideal opamp gain
  \end{enumerate}
  \item For an ideal opamp, what do we want of the
  \begin{enumerate}
    \item Input impedance
    \item Output impedance
  \end{enumerate}
\end{enumerate}