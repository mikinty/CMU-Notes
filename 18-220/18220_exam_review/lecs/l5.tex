\subsection*{Lecture 5}
Lab1, capacitors, parasitic capacitance in MOSFETs and effect on high
frequency action, Thevenin and Norton equivalents.

\begin{enumerate}
  \item Why is the Manchester Encoding useful?
  \item Why does your MOSFET have a frequency limit at which it can operate at?
  \item Know the parallel and series capacitance formulas.
  \item What is the equivalent capacitance of the following circuit?
  \begin{figure}[H]
    \centering
    \begin{circuitikz} 
      \draw 
        (-1, 0) to[short, o-]
        (0, 0) to[C=$C_1$] (2, 0)
        to[C=$C_2$] (2, -2)
        to[C=$C_3$] (2, -4)
        to[C=$C_4$] (0, -4)
        to[C=$C_5$] (0, -2)
        to[short] (0, -1)
        to[short, -o] (-1, -1)
        (0, -2) to[C=$C_6$] (2, -2)
      ;
    \end{circuitikz} 
  \end{figure}
  \item What are Thevenin and Norton equivalents? Why do we use them?
  \item Find the Thevenin and Norton equivalent for the following circuit,
  \footnote{Do it symbolically, it will make you a stronger problem solver.
  Otherwise, plug in some numbers for yourself}
  \begin{figure}[H]
    \centering
    \begin{circuitikz}
      \draw
        (0, 0) to[R=$R_1$] (2, 0)
        to[R=$R_2$] (2, -2)
        to[short] (0, -2)
        to[american voltage source, l=$V_1$, invert]
        (0, 0)
        (2, -2) to[american current source, l=$I_1$]
        (2, -4) to[short, -o] (5, -4)
        (4, -4) to[R=$R_4$] (4, -2)
        to[short] (2, -2)
        (4, -2) to[R=$R_3$]
        (4, 0)
        (2, 0) to[short, -o] (5, 0)
      ;
    \end{circuitikz}
  \end{figure}
  \item Why are nonlinear equations hard to solve? (Don't just say nonlinear
  means things are complicated, try to give a more specific reason. If you
  need a hint, think about what techniques you use to solve linear equations)
  \item Know the charge and energy equations for a capacitor.
\end{enumerate}

\subsection*{Equations}
Capacitor energy equations,
\begin{equation*}
  E = \frac{C \cdot V^2}{2}, Q = C \cdot V
\end{equation*}

Thevenin and Norton relation
\begin{equation*}
  V_\text{T} = I_\text{N} \cdot R_\text{EQ}
\end{equation*}