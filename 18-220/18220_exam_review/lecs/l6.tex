\subsection*{Lecture 6}
Power, RC circuits.

\begin{enumerate}
  \item What are initial conditions of a circuit, and why do we care about
  them when we solve for differential equations?
  \item If you don't by now, you should definitely know how to use KVL, KCL
  in circuit problems.
  \item Solve the RC step problem,
  \begin{figure}[H]
    \centering
    \begin{circuitikz} 
      \draw 
        (0, 0) node[sground] {}
        to[american voltage source, invert]
        (0, 2) to[R=$R$] (2, 2)
        to[C=$C$] (2, 0)
        node[sground] {}
      ;

      \draw
        (-1.5, 0.5) node[left] {0}
        -- (-1, 0.5) -- (-1, 1.5) -- (-0.5, 1.5)
        (-1.5, 1.5) node[left] {$V_1$}
      ;
    \end{circuitikz} 
  \end{figure}
  \item We believe in the conservation of energy, but in real life, if you
  have an oscillation circuit, e.g. LC, it will eventually stop oscillating.
  Where is the energy ``conserved''?
  \item What does a capacitor behave like when it is fully charged/discharged?
\end{enumerate}
