\subsection*{Lecture 9}
Nodal analysis. MNA is basically just setting up a system of linear equations.

\begin{enumerate}
  \item If you don't already know, if you have $N$ unknowns, you need $N$
  linearly independent equations to solve for these $N$
  variables.\footnote{Linearly independent means you can't derive any of the
  equations from the other set of the equations. At a high level, a linear
  independent equation is giving us more, unique information.}
  \item Why are the following circuits illegal?
  \begin{itemize}
    \item Circuit 1
    \begin{figure}[H]
      \centering
      \begin{circuitikz}
        \draw (0, 0) 
          to[american voltage source, V=$\SI{1}{\volt}$, invert]
          (0, 2) to[short] (2, 2) 
          to[american voltage source, V=$\SI{2}{\volt}$]
          (2, 0) -- (0,0)
        ;
      \end{circuitikz}
    \end{figure}
    \item Circuit 2
    \begin{figure}[H]
      \centering
      \begin{circuitikz}
        \draw (0, 0) 
          to[american current source, l=$\SI{1}{\ampere}$]
          (0, 2) 
          to[american current source, l=$\SI{2}{\ampere}$]
          (0, 4)
        ;
      \end{circuitikz}
    \end{figure}
  \end{itemize} 
  \item When we say there is ``no true ground,'' what do we mean by that?
  How can you use this to your advantage when solving MNA problems?
  \item What special steps do we have to take in MNA for the following elements?
  \begin{itemize}
    \item Voltage source
    \item Current source
    \item Capacitor
    \item Inductor
  \end{itemize}
  \item Understand how to use superposition to solve circuit problems.
\end{enumerate}

\subsection*{Equations}
MNA Steps:
\begin{enumerate}
  \item Choose a ground 
  \item Fill in known values 
  \item Write dependent equations for special elements
  \item Write full set of equation with KCL
\end{enumerate}