\subsection*{Lecture 2}
Introduction to how we view circuits, abstractions and modeling.

You learn about op-amps, what they are used for, and how to solve problems
with them in the ideal and non-ideal contexts.

\begin{enumerate}
  \item What are the assumptions for an ideal op-amp? Also mention that
  implications they have to help us solve opamp circuit problems.
  \item What is the transfer function ($\vo/\vi$) for an inverting op-amp?
  \begin{figure}[H]
    \centering
    \begin{circuitikz} 
      \draw 
        (0, 0) node[op amp] (opamp) {}
      ;

      \draw 
        (opamp.-) to[R=$R_1$] ++(-1, 0)
          node[left] {$\vi$}
        (opamp.-) -- ++(0, 1)
          to[R=$R_2$] ($(opamp.out) + (0, 1.5)$)
          -- (opamp.out)
        (opamp.out) -- ++(0.5, 0) node[right] {$\vo$}
      ;

      \draw (opamp.+) node[sground] {};
    \end{circuitikz}
    \label{l2:opamp_inverting}
  \end{figure} 

  \item What happens when our opamp is not ideal, and has 
  \begin{itemize}
    \item Nonzero input current?
    \item Finite input resistance?
    \item Finite output impedance?
  \end{itemize}

  \item What is voltage/current gain?
  \item Derive the transfer function for an inverting opamp with finite gain.
\end{enumerate}

\subsection*{Equations}
\begin{equation}
  V = IR
\end{equation}
Ohm's law, $V$ is the voltage, $I$ is the current and $R$ is the resistance.
Probably the most important fundamental equation used to solve circuit problems.
\begin{equation}
  \lambda = \frac{c}{f},
\end{equation}
where $\lambda$ is the wavelength, $f$ is the frequency,
and $c = \SI{3e8}{\m/\s}$.