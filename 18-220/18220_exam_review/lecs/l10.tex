\subsection*{Lecture 10}
LC, RLC circuits, switching power converters.

\begin{enumerate}
  \item We can use a voltage divider to create any voltage $\in [0, V]$ from
  a voltage source with maximum voltage $V$. Why do we bother to use
  switching power converters?
  \item How does a switching power converter work? What are the steps, and
  what components do we need at each step?
  \item Draw the following converters, and derive the equations that relate
  the input voltage, duty cycle, and output voltage.\footnote{Even though you
  can just memorize the final equations, you are sometimes asked to derive
  them on exams.}
  \begin{itemize}
    \item Buck
    \item Boost
    \item Flyback (Buck-Boost)
  \end{itemize}
  \item Draw an LC oscillator. How does it behave? Write down the
  differential equation that dictates its operation.
  \item In this class, the steps for solving a differential equation are 
  \begin{enumerate}[label=\arabic*.]
    \item Homogeneous solution, we usually assume in the form $Ae^{st}$
    \item Particular solution
    \item Initial Conditions
  \end{enumerate}
  It's a good idea to format your solutions to problems like this so it's
  easier for you and the grader.
  \item Draw an RLC oscillator (there are multiple forms). How is it
	different from an LC circuit? Derive the differential equation for an RLC circuit.
	\item Draw a overdamped, critically damped, and underdamped time response for an RLC circuit.
\end{enumerate}

\subsection*{Equations}
Buck converter
\begin{figure}[H]
	\centering
	\begin{circuitikz}
		\draw (0, 0) node[sground] {}
		to[V=$\Vin$, invert] (0,2)
		to[nos, l=D] (2,2)
		to[D*, invert] (2,0) node[sground] {}
		;

		\draw (2,2)
		to[L=$L$, i=$I_L$] (5,2) node[right] {$\Vout$}
		to[R=$R_L$] (5,0) node[sground] {}
		;
	\end{circuitikz}
\end{figure}
\begin{equation*}
  \frac{\Vout}{\Vin} = D
\end{equation*}
Boost converter
\begin{figure}[H]
	\centering
	\begin{circuitikz}
		\draw (0, 0) node[sground] {}
		to[V=$\Vin$, invert] (0,2)
		to[L] (2,2)
		to[nos] (2,0) node[sground] {}
		;

		\draw (2,2)
		to[D*] (4, 2) -- (6,2) node[right] {$\Vout$}
    to[R=$R_L$] (6,0) node[sground] {}
    (4,2) to[C] (4, 0) node[sground] {}
		;
	\end{circuitikz}
\end{figure}
\begin{equation*}
  \frac{\Vout}{\Vin} = \frac{1}{1-D}
\end{equation*}
Flyback (Buck-Boost)
\begin{figure}[H]
	\centering
	\begin{circuitikz}
		\draw (0, 0) node[sground] {}
		to[V=$\Vin$, invert] (0,2)
		to[nos] (2,2)
		to[L] (2,0) node[sground] {}
		;

		\draw (2,2)
		to[D*, invert] (4, 2) -- (6,2) node[right] {$\Vout$}
    to[R=$R_L$] (6,0) node[sground] {}
    (4,2) to[C] (4, 0) node[sground] {}
		;
	\end{circuitikz}
\end{figure}
\begin{equation*}
  \frac{\Vout}{\Vin} = \frac{-D}{1-D}
\end{equation*}