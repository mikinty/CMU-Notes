\subsection*{Formulas}
\begin{itemize}
  \item Ohm's law
  \begin{equation}
    V = IR
    \label{eq:vir}
  \end{equation}
  $V$ is the voltage, $I$ is the current and $R$ is the resistance. Probably
  the most important fundamental equation used to solve circuit problems.

  \item Power
  \begin{equation}
    P = VI = I^2R = \frac{V^2}{R}
  \end{equation}

  \item Thevenin and Norton relation
  \begin{equation}
    V_\text{T} = I_\text{N} \cdot R_\text{EQ}
    \label{eq:thev_nort}
  \end{equation}

  \item MNA Steps:
  \begin{enumerate}
    \item Choose a ground 
    \item Fill in known values 
    \item Write dependent equations for special elements
    \item Write full set of equation with KCL
  \end{enumerate}

  \item Wave equation
  \begin{equation}
    \lambda = \frac{c}{f},
    \label{eq:wave_eq}
  \end{equation}
  where $\lambda$ is the wavelength, $f$ is the frequency,
  and $c = \SI{3e8}{\m/\s}$.

  \item Diode equation
  \begin{equation}
    \iD = \IS \pa{e^{\vD/\VT} - 1}
    \label{eq:diode_eq}
  \end{equation}

  \item The N-MOSFET equations are 
  \begin{align*}
    \ID &= \frac{1}{2}\Kn \pa{\VGS - \VTH}^2 \tag{Saturation, $\VDS > \VGS - \VTH$}\\
    \ID &= \frac{1}{2}\Kn \pa{2\pa{\VGS - \VTH}\VDS - \VDS^2} \tag{Triode, $\VDS < \VGS - \VTH$}
  \end{align*}
  the current is zero when the MOSFET is off, when $\VGS < \VTH$.

  \item Capacitor equation
  \begin{equation}
    i_C(t) = C \dv{V_C(t)}{t}
    \label{eq:cap}
  \end{equation}

  \item Capacitor energy equations
  \begin{equation}
    E_L = \frac{1}{2}C V_C^2, Q = C \cdot V
    \label{eq:cap_energy}
  \end{equation}

  \item Inductor equation
  \begin{equation}
    v_L(t) = L \dv{i_L(t)}{t}
    \label{eq:ind}
  \end{equation}

  \item Inductor energy equations 
  \begin{equation}
    E_L = \frac{1}{2}LI_L^2
    \label{eq:ind_energy}
  \end{equation}

  \item 
  Transformer equivalent impedance
  \begin{figure}[H]
    \def\offsetX{6}
    \def\offsetY{-1}
    \centering
    \begin{circuitikz} 
      \draw (0, 0) node [transformer core](T){};
      \draw
        (T.A1) to[R=$R_1$] ++(-2, 0)
        to[american voltage source, l_=$V_1$]
        ($(T.A2) + (-2, 0)$) node[sground] {}
        -- (T.A2)
        (T.B1) to[short] ++(2, 0)
        to[R=$R_2$] 
        ($(T.B2) + (2, 0)$)
        -- (T.B2)
      ;

      \draw
        (4.5, 0) node {$\Rightarrow$}
      ;

      \draw
        (\offsetX, \offsetY) node[sground] {}
        to[american voltage source, V=$V_1$, invert]
        ++(0, 2) to[R=$R_1$] ++(2, 0)
        to[R=$\pa{\frac{N_1}{N_2}}^2R_2$]
        ++(0, -2) -- (\offsetX, \offsetY)
      ;
    \end{circuitikz} 
  \end{figure}

  \item Buck converter
  \begin{figure}[H]
    \centering
    \begin{circuitikz}
      \draw (0, 0) node[sground] {}
      to[V=$\Vin$, invert] (0,2)
      to[nos, l=D] (2,2)
      to[D*, invert] (2,0) node[sground] {}
      ;

      \draw (2,2)
      to[L=$L$, i=$I_L$] (5,2) node[right] {$\Vout$}
      to[R=$R_L$] (5,0) node[sground] {}
      ;
    \end{circuitikz}
  \end{figure}
  \begin{equation}
    \frac{\Vout}{\Vin} = D
    \label{eq:buck}
  \end{equation}

  \item Boost converter
  \begin{figure}[H]
    \centering
    \begin{circuitikz}
      \draw (0, 0) node[sground] {}
      to[V=$\Vin$, invert] (0,2)
      to[L] (2,2)
      to[nos] (2,0) node[sground] {}
      ;

      \draw (2,2)
      to[D*] (4, 2) -- (6,2) node[right] {$\Vout$}
      to[R=$R_L$] (6,0) node[sground] {}
      (4,2) to[C] (4, 0) node[sground] {}
      ;
    \end{circuitikz}
  \end{figure}
  \begin{equation}
    \frac{\Vout}{\Vin} = \frac{1}{1-D}
    \label{eq:boost}
  \end{equation}

  \item Flyback (Buck-Boost)
  \begin{figure}[H]
    \centering
    \begin{circuitikz}
      \draw (0, 0) node[sground] {}
      to[V=$\Vin$, invert] (0,2)
      to[nos] (2,2)
      to[L] (2,0) node[sground] {}
      ;

      \draw (2,2)
      to[D*, invert] (4, 2) -- (6,2) node[right] {$\Vout$}
      to[R=$R_L$] (6,0) node[sground] {}
      (4,2) to[C] (4, 0) node[sground] {}
      ;
    \end{circuitikz}
  \end{figure}
  \begin{equation}
    \frac{\Vout}{\Vin} = \frac{-D}{1-D}
    \label{eq:flyback}
  \end{equation}

  \item Resonant frequency of LC circuit
  \begin{equation}
    \omega = \frac{1}{\sqrt{LC}}
    \label{eq:lc}
  \end{equation}
  
  \item Antenna equations
  \begin{figure}[H]
    \def\labelLambdaHeight{0.35}
    \def\labelLengthHeight{0.7}
    \def\antennaWidth{3}

    \centering
    \begin{circuitikz}
      % Antenna Length
      \draw[] (-\antennaWidth, 0) -- (\antennaWidth,0);

      % Antenna bottom stick
      \draw[] (0,0) -- (0,-0.5);
      \draw[fill] (0,0) circle (5pt);

      % Labels
      \draw[<->] (-\antennaWidth, \labelLambdaHeight) -- (0, \labelLambdaHeight) node[fill=white, midway] {$\lambda/4$};
      \draw[<->] (0, \labelLambdaHeight) -- (\antennaWidth, \labelLambdaHeight) node[fill=white, midway] {$\lambda/4$};
      \draw[<->] (-\antennaWidth, \labelLengthHeight) -- (\antennaWidth, \labelLengthHeight) node[fill=white, midway] {$L$};
    \end{circuitikz}
    \caption{Half-wave dipole antenna}
  \end{figure}
  \begin{equation}
    L = \frac{1}{2}\lambda
    \label{eq:antenna}
  \end{equation}

  \item Coaxial cable model
    \begin{figure}[H]
      \centering
      \begin{circuitikz}
        \draw (0,0)
        to[R=$R$, o-] (2,0)
        to[L=$L$] (3,0) 
        to[short, -*] (4.5,0)
        to[short, -o] (5.5,0)
        ;

        \draw (0,-2)
        to[short, o-*] (4.5,-2)
        to[short, -o] (5.5,-2)
        ;

        \draw (4.5,0)
        to[R=$G'$] (4.5,-2)
        ;

        \draw (3.5,0)
        to[C, l_=$C'$] (3.5,-2)
        ;
      \end{circuitikz}
      \caption{Model for a Coaxial Cable}
    \end{figure}

  \item Reflection coefficient for coaxial cable 
  \begin{equation}
    \Gamma = \frac{Z_L - Z_S}{Z_L + Z_S},
    \label{eq:refl}
  \end{equation}
  where $Z_L$ is the impedance of the load and $Z_S$ is the characteristic
  impedance.

  \item Hertz to radians
  \begin{equation}
    f = \frac{\omega}{2 \pi}
    \label{eq:hertz_radians}
  \end{equation}

  \item Decibel to absolute gain relation
  \begin{equation}
    \text{Gain}\pa{\text{dB}} = 20\log_{10}\frac{\Vout}{\Vin}
    \label{eq:gaindb}
  \end{equation}

  \item Canonical transfer function
  \begin{equation}
    \frac{\Vout}{\Vin} = \frac{
      -A \jw/\omega_\text{z}
    }{
      (1 + \jw/\omega_1)(1 + \jw/\omega_2)
    }
    \label{eq:tfcanon}
  \end{equation}
\end{itemize}