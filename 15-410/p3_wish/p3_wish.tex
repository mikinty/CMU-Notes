\documentclass{article}
\usepackage[utf8]{inputenc}
\usepackage{cs_hw}

\pagestyle{fancy}

\begin{document}\thispagestyle{empty} % No header on first page
\maketitle
\te{p3} is the giant of the 15-410 experience, and one of the largest projects you'll ever do at CMU.
While the assignment's individual parts aren't that hard, choosing the wrong design, and making
poor choices throughout the project can lead to devastating outcomes, since large amounts of code
is very difficult to fix, and mistakes are amplified over time as you do the project.

The project handout from the 15-410 course staff compiles some warnings from previous
semester, but I feel like there can be a bit more detail for certain parts. I will 
detail some parts of the handout I think deserved more attention, and hopefully it will
help you as well when you undertake the \te{p3} experience.

\section{Relationship between \te{p2} and \te{p3}}


\section{Error Checking}
\label{sec:error}
In class, they introduce why error checking is important and how you should
go about handling errors in your code. The main lesson learned from class however,
is that 
\begin{equation}
  \te{No error should be left unheard}.
\end{equation}
That makes sense, and what this means to a lot of people is just to implement
the pass-the-puck error handling approach to everything.

However, the side that is not taught as much is how to actually handle these
error, and how to write assertions throughout your code to guarantee 
that it is more safe. 

Below, I'll outline the 3 most important error checking principles you should
utilize throughout your project.

\subsection{Passing the Puck}
As mentioned earlier, this is usually the easiest to understand method of handling
errors. The idea is that whenever something is wrong, you just return a negative error code.
Since there can be many different errors, you should have different negative error codes
for different errors. One approach you can take is to have a bunch of defined
error codes that are shared by functions throughout your kernel.

\begin{lstlisting}[caption={Negative error code defines}]
  #define FUNCTION_ERRNO_NULL -1
  #define FUNCTION_ERRNO_INIT -2
  #define FUNCTION_ERRNO_ARGS -3
\end{lstlisting}

\subsection{Assertions}
While passing the puck is easy to implement, it is very weak, since returning a negative error code means you won't
be able to pinpoint where the error is until you have code that checks for a negative error code.
For example, you could potentially have a mutex init that fails, which gets passed up to a pcb init,
which then fails and is passed up to the kernel init function. You'll know by the kernel init 
function that something went wrong, but you'll have to track down some more to figure out when things
\textit{really} went wrong.

A more specific error check you can do is an assertion, which course staff sets 
up for you in \te{contracts.h}. Assertions are useful because when something goes wrong,
the exact line number will be returned to you at the time of error. If you combined
assertions with a good code tracing utility, you can debug your errors much faster.
In addition, assertions are good when you have an error that you can't really recover from.

A good way to write your kernel is to write a bunch of assertions that 
describe how you expect your function to behave, and fix things when
assertions start to break. This is actually a common practice in industry, 
known as 
\href{https://en.wikipedia.org/wiki/Test-driven_development}{TDD (Test Driven Development)}.
While this method of development can be rather slow, it can help you write much more
robust code, which is important for a project of this scale.

As an example, consider the following functions, one written entirely with passing the puck,
and the other with a mix in Listing \ref{lst:error_style}.
The \te{assertStyle} code is preferred here, since unrecoverable errors will
trigger the assertion error, helping you pinpoint errors, while errors that are
not as severe, for example passing in a bad value for a argument, are not passed
to the caller to handle. 

\begin{lstlisting}[label={lst:error_style}, caption={Comparing passing the puck and assertion styles}]
  function puckStyle (int *apples, int pineapple) {
    if (apples == NULL) {
      return FUNCTION_ERRNO_NULL;
    }

    if (valid_mem(apples)) {
      return FUNCTION_ERRNO_MEM;
    }

    if (pineapple == 0) {
      return FUNCTION_ERRNO_ARGS;
    }

    apples += pineapple;

    return pineapple;
  }

  function assertStyle (int *apples, int pineapple) {
    ASSERT(apples != NULL);
    ASSERT(valid_mem(apples));

    if (pineapple == 0) {
      return FUNCTION_ERRNO_ARGS;
    }

    apples += pineapple;

    return pineapple;
  }
\end{lstlisting}

\subsection{\te{panic}}
I didn't do a whole lot of debugging in \te{p2}, so I didn't use panic.
When I got to \te{p3}, I didn't really know what it was for.
In the handout, we are told that \te{panic} is way to help us debug our code,
and a good \te{panic} function will help us debug faster. When I first heard that,
I had no idea what that meant...sounded like empty words to me.

The reason course staff is pushing you to write a \te{panic} function is because
of the following -- have you ever had the following scenario?
\begin{itemize}
  \item You run your program, and there's a bug
  \item You stop the execution, and look around the context
  \item Let me look at \te{\%esp, \%eip, \%cs, \%ss}... 
  \item Ok hmm...things look alright. Let me check the contents of the stack
  \item That seems fine...let me look at the parameters of the current TCB
  \item Alright, I'm curious what tasks are currently scheduled in the scheduler. 
  Ugh...this is going to be a lot of \te{simics psym} commands. 
  \item Oof, I also want to see what threads are current sleeping.
  \item Hmm...would also be nice to know how many free pages I current have.
\end{itemize}
The above debugging journey is not necessarily hard to do, but it is certainly \textit{time consuming}.
Typing in \te{simics} commands is quite unwieldy, and is also limited in what you can do.
A \te{panic} function can help you do all these things in one go. Something goes wrong -- \te{panic}!
Everything you want to know in general can be printed out, from the current task, to the tasks in the scheduler,
the current timer stats, the context on the surrounding stack area -- literally anything you deem useful.
Since your error-checking code, like \te{ASSERT}, will call \te{panic}, every time something goes wrong,
you'll automatically be able to get a printout of what happens. You can even implement checks in your \te{panic}
to automatically check certain sanity things for you, for example if your segment registers make sense or 
the program instruction pointers are reasonable.  

Some features you might consider putting into your \te{panic}:
\begin{itemize}
  \item Print out the current state
  \begin{itemize}
    \item Registers
    \item Any state variables stored in the kernel
    \item Surrounding stack near the \te{\%esp}
    \item Surrounding stack of the user program, if applicable
  \end{itemize}
  \item Print out the current TCB 
  \item Print out the current PCB
  \item Print out some tasks in the scheduler's active, sleep queues
  \item Print out global variables
\end{itemize}

\section{Locking}

\section{Plan of Attack}
There is a plan of attack section listed at the end of the handout that gives
a nice overview about what steps you should approximately take to complete the project.

\begin{enumerate}
  \item Strive to document your code as you go. At the very least,
  make \te{doxygen} headers on all the functions and files you create,
  so you don't have to go through the pain of adding them in later on.

  \item One of the first things you should do is write all your system call and interrupt handlers.
  This includes exception handlers! They don't have to necessarily work yet (you can just \te{return} for now),
  However, if you're missing a handler and it gets triggered for whatever
  reason, you will get a mysterious \te{General Protection Fault}, since the handler is not installed
  in the IDT.

  You will probably find macros useful for installing all your interrupt handlers, since 
  many handlers will share many parts. Early on though, you might want to just write it all out
  just because it's easier to do.

  \item When they say to comment out some of your \te{install\_autostack} code,
  what you should understand is that for any program, the first function that is called
  is the function in \te{crt0.c}, which just performs the \te{install\_autostack} and
  then calls \te{exit} into the program you intended to execute. Therefore, since 
  \te{install\_autostack} will be called by every program, and the function requires syscalls
  you may not have written yet, i.e. \te{new\_pages}, you will likely need to comment out most
  if not all of \te{install\_autostack}.

  \item They tell you to choose an error-handling approach early. When I originally
  read this, I thought they just meant to return various types of error codes.
  What I realized in retrospect was that you can use error checking to your advantage as you
  develop. See Section \ref{sec:error} for more details about how to use error checking
  as you write \te{p3}.

  \item If you are noticing that you are running a lot of the tests over and over again...especially
  the hurdle tests, it might be a good idea to write a program that runs suites of tests for you automatically.
  It'll save you time, and also encourage you to test more. One good practice when your kernel starts
  getting big is to run a set of tests every time you make a change, so if your change breaks something,
  you'll immediately know.
\end{enumerate}

\end{document}